\begin{tikzpicture}
	\path[
		mindmap,
		concept color=purple!80,
		text=black
	]
		node [concept, font=\huge] {Minőség}
		[clockwise from=0, level 1 concept/.append style={sibling angle=180}]
		child[concept color=teal!50] {
			node[concept, font=\large] (usability) {Hasz\-nál\-ha\-tó\-ság}
			[clockwise from=135, level 2 concept/.append style={sibling angle=45, visible on=<3->}]
			child { node[concept] {absztrakció} }
			child { node[concept] {hibák vonzása} }
			child { node[concept] {kon\-zisz\-tens\-ség} }
			child { node[concept] {leképezés közelisége} }
			child { node[concept] {szellemi erőfeszítés} }
			child { node[concept] {ter\-jen\-gős\-ség} }
			child { node[concept] {viszkozitás} }
		}
		child[concept color=pink] {
			node[concept, font=\large] {Hasznosság}
			[counterclockwise from=90, level 2 concept/.append style={sibling angle=60, visible on=<2->}]
			child { node[concept] (expressiveness) {ki\-fe\-je\-ző\-e\-rő} }
			child { node[concept] {ro\-bosz\-tus\-ság} }
			child { node[concept] {ember\-orien\-táltság} }
			child { node[concept] {feladat\-orien\-táltság} }
		};

		\node[below=3 of usability, visible on=<3->] {Cognitive dimensions of notations (Thomas Green, 1989)};
		%\node[above of=expressiveness, visible on=<2->] {(Mathias Felleisen, 1991)};
\end{tikzpicture}
