

A PLanG\footnote{%
	A betűszó feloldása Lövei László szíves közlése szerint egyszerűen \textit{Programming Lan\-guage}, de érdemes megemlíteni az egyetemen elterjedt legendát, hogy a PLanG a \textit{Pázmány Language} rövidítése.}%
programozási nyelvet 2005 óta használják karunkon a programozás alapjainak oktatására: a nyelvet és a hozzá tartozó fejlesztői környezetet direkt erre a célra tervezte és implementálta Lövei László.

A PLanG oktatási célú nyelv, tervezése során az elsődleges szempont a procedurális programok működésének demonstrálása volt\cite{lovei}, ezért eszköztára meglehetősen szegényes: fájdalmasan hiányoznak az alprogramok, a módszertan oktatását segítő eszközök, de még az alapvetőnek tekinthető \textit{elsif} kon\-strukció is hiányzik a nyelvből.
A nyelv Turing-teljes ugyan (elvégre rendelkezik elágazással és ciklussal), így elvben minden fejlettebb programozói eszköz szimulálható rajta, de ez -- alprogramok híján -- csak nagyon nehézkesen lehetséges, a programozással épp csak ismerkedő diákoktól semmiképp nem várható el.
Oktatási célú programozási nyelvként a PLanGnak egyszerűen használható, kiterjedt programozói eszköztárat nyújtva inspirálnia kellene a diákokat, megnyitni előttük a programozás világát.
Ehhez képest azt tapasztaljuk, hogy a PLanG gyakran gátolja a tanulót a programozás alapelveinek elsajátításában.

Önálló laboratóriumi feladatom az volt, hogy ezt az intuitív meglátást racionális elvekkel támasszam alá: formális rendszert keresve értékeljem a PLanG minőségét, illetve vázoljak fel egy lehetséges kiterjesztést, ami javítaná a nyelv minőségét.

\Aref{sec:proglang}. részben minőségi mutatókat keresünk a programozási nyelvekhez, különös tekintettel az oktatási célú programozási nyelvekre.

\Aref{sec:fordprog}. részben futólag áttekintjük a fordítóprogramok általános működését.

\Aref{sec:xplang}. részben \aref{sec:proglang}. részben ismertetett metrikák szerint vizsgálom a PLanGot és javítási lehetőségeit, és beszámolok a PLanG kiterjesztett változatát fordító program prototípusáról.
