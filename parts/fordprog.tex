
Formálisan a fordítóprogram egy $T: \mathbb{P}_\mathcal{L} \rightarrow \mathbb{P}_{\mathcal{L}'}$ transzformációs függvény, melyre $\forall P \in \mathbb{P}_\mathcal{L}: eval_\mathcal{L}(P) \Rightarrow eval_{\mathcal{L}'}(T(P))$.
$\mathcal{L}$-t \textbf{forrásnyelv}nek, $\mathcal{L}'$-t pedig \textbf{tárgynyelv}nek nevezzük.
A forrásnyelv általában magasszintű, a tárgynyelv pedig alacsonyabb szintű, például assembly-szintű, vagy gépi kódú nyelv: az ilyen átalakítást végző programot \textit{compiler}nek nevezzük\cite{Csornyei}.

A compiler feladata a gyakorlatban az, hogy a forrásnyelvi programot az embereknek szánt, szöveges formából a gép számára futtatható programmá alakítsa.

\subsection{A compiler általános felépítése}
A compiler feladatát két jól elkülöníthető részfeladatra bontjuk: az \textbf{analízis} fázisában a forrásnyelvi programot elemzi, és az arről nyert tudást a program egy \textbf{belső reprezenzáció}nak nevezett struktúrában tárolja; majd a \textbf{szintézis} fázisában a belső reprezentáció alapján felépíti a tárgynyelvi programot\cite{Csornyei}\cite{torczon_cooper}.

\begin{figure}[htb]
	\centering
	\resizebox{\textwidth}{!}{\begin{tikzpicture}[
	clas/.style={rectangle, rounded corners, draw, fill=blue!10, inner sep=4pt, text width=6em, text centered, minimum height=2em, font=\bfseries\sffamily\small},
	arr/.style={->, >=latex', shorten >=1pt}
	]

\node (input) {};
\node[clas] (lexer) [right=of input] {Lexikális elemző};
\node[clas] (parser) [right=of lexer] {Szintaktikus elemző};
\node[clas] (semantics) [right=of parser] {Szemantikus elemző};

\node[clas] (codegen) [right=3 of semantics] {Kódgenerátor};
\node[clas] (optimize) [right=of codegen] {Optimalizáló};
\node (output) [right=of optimize] {};
%\node (error-out) [above=of parser] {};

\begin{scope}[on background layer]
	\node[
		rectangle, rounded corners, draw, fill=blue!10, fill opacity=0.4, inner sep=0.8em,
		label={below:{\bfseries\sffamily Analízis}},
		fit=(lexer) (parser) (semantics)
	] (analysis){};

	\node[
		rectangle, rounded corners, draw, fill=blue!10, fill opacity=0.4, inner sep=0.8em,
		label={below:{\bfseries\sffamily Szintézis}},
		fit=(codegen) (optimize)
	] (synthesis){};
\end{scope}

\path[arr, every node/.style={font=\itshape\footnotesize,
  		fill=white,inner sep=3pt}]
(input) edge (lexer)
(lexer) edge (parser)
(parser) edge (semantics)
(semantics) edge node[above] {Belső} node[below] {reprezentáció} (codegen)
(codegen) edge (optimize)
(optimize) edge (output);

\end{tikzpicture}
}
	\caption{A compiler felépítése.}
\end{figure}

Az analízis első feladata, hogy a forrásnyelvi programot reprezentáló karaktersorozatot a nyelv lexikális elemeit reprezentáló szimbólumok sorozatára bontsa.
Ezt a feladatot a \textit{lexikális elemző} végzi.
A kapott szimbólumsorozatot a \textit{szintaktikus elemző} vizsgálja tovább, a szekvenciális mondatból próbálja felállítani a program hierarchikus struktúráját: azt a kérdést próbálja megválaszolni, hogy egy formális nyelvtan milyen levezetése adja ki a kapott mondatot.
Ha a program szintaktikailag helyesnek bizonyul, akkor tovább elemzi a \textit{szemantikus elemző}, amelynek feladata, hogy ellenőrizze, megfelel-e a program a statikus szemantika szabályainak.

Az analízis kimenete a belső reprezenztáció, amely a forrásnyelvi programmal szemantikailag ekvivalens \textit{köztesnyelv}i program.
A belső reprezentáció a gyakorlatban sokféle lehet: valamilyen fastruktúra, valamilyen absztrakt utasítások szekvenciális sorozata.\cite{torczon_cooper}

A szintézis során a \textit{kódgenerátor} a belső reprezentáció minden egységéhez szemantikailag ekvivalens tárgynyelvi utasítást rendel.
A kapott tárgynyelvi programon a \textit{kódoptimalizáló} további transzformációkat végez, hogy növelje a kapott program hatékonyságát.


\subsection{A környezetfüggetlen analízis eszközei}
Miután a lexikális elemző a program szövegét a forrásnyelv felszíni szintaxisát leíró $G$ környezetfüggetlen nyelvtan terminális szimbólumainak $s$ sorozatává, a nyelv \textit{mondat}ává, alakította, a szintaktikus elemző feladata konstruktívan bizonyítani, hogy $s$ valóban levezethető $G$-ből.
A bizonyítás konstruktív volta az jelenti, hogy az elemző algoritmus ténylegesen megadja az $s$-t generáló levezetési fát.
Egy nyelvtant \textit{nemegyértelmű}nek mondunk, ha létezik hozzá olyan mondat, amely többféleképpen levezethető, ami avval jár, hogy különböző elemzésekhez különböző tárgyprogramok tartoznak -- ez nem kívánatos tulajdonság.

A környezetfüggetlen analízis jól kutatott, megoldottnak tekinthető probléma\footnote{
	Ami nem jelenti, hogy ne lenne fejlődés, ne születnének máig újabb, hatékonyabb algoritmusok, megközelítések: például Bryan Ford 2004-ben ismertette a Chomsky-nyelvosztályokból kilépő \textit{Parsing Expression Grammar} nevű nyelvosztályt és az ezen alapuló szintaktikai elemzőt\cite{Ford04PEG}.
}.
Early 1968-ban adott algoritmust, amely tetszőleges környezetfüggetlen nyelvtant $\mathcal{O}(n^3)$ időben elemez, azonban sok algoritmus létezik, amely a környezetfüggetlen nyelvek halmazának valamely részhalmazába tartozó nyelveket ennél hatékonyabban, vagy egyszerűbben implementálható módon elemzi.
A legfontosabb ilyen nyelvosztályok az $LR(1)$ grammatikák, az $LL(k)$ és az $LL(1)$ grammatikák osztálya.
Az ezen nyelvosztályokat meghatározó algoritmusokkal könnyen automatizálható módon \textit{elemző táblázat}okat építhetünk a nyelvhez, amelyek lineáris időben képesek feldolgozni a bemeneti karaktersort\footnote{%
	Ezen és egyéb elemző algoritmusok leírásáért lásd \cite[4-6.~fejezetek]{Csornyei}.
}.
Az ilyen elemző táblázatokat építő automatizáló eszközöket az iparban széles körűen használják.
Ilyen például a híres \textit{YACC}, illetve nyílt forráskódú változata, a \textit{GNU bison} (módosított $LR(1)$ elemzők), és a modernebb, Java-alapú \textit{ANTLR} (módosított $LL(k)$ elemző).

%\subsubsection{Rekurzív leszállásos elemzés}
Az automatizált elemzőgenerátorok hasznos eszközök, de a fordítóprogramok tervezői gyakran mégis kézzel írják meg az elemzőt.
Régi, elterjedt, bevált módszer a \textit{rekurzív leszállásos elemzés} módszere, melynek lényege, hogy a nyelvtan minden nemterminális szimbólumához a szimbúlum helyettesítési szabályát mgvalósító eljárást rendelve, egy vagy több előreolvasott szimbólum alapján döntést hozva, az implementációs programozási nyelv végrehajtási vermét használva szimulálunk veremautomatát.

A rekurzív leszállásos elemező előnye a táblázatos módszerekkel szemben, hogy nagyon könnyű megvalósítani, lehetőséget ad informatív hibaüzenetek generálására, és jól olvasható is, amennyiben az elemzőt végignézve megismerhetjük a nyelvtant.
További előny, hogy az eljárásokban megkötés nélkül ki tudjuk használni a Turing-teljes implementációs nyelv teljes eszközkészletét, vagyis a szimulált veremautomatánk kifejezőerejét tetszés szerint növelhetjük -- igaz, az implementációs komplexitás kárára.
Rekurzív leszállásos elemzőt használ például a Lua programozási nyelv fordítóprogramja.



% operator problem, landin?, pratt
