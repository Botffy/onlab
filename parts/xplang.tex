
Mint minden szoftverfolyamatnak, a nyelvtervezés első fázisa is a követelményfeltárás: a nyelv feladatterének elemzése, a felhasználói igények felmérése.
Mivel létező nyelv kiterjesztéséről van szó, elemezni kellett a létező nyelvet is \aref{subsubsec:edulang}. részben ismertetett metrikák szerint.

\subsection{A PLanG programozási nyelv értékelése}
A PLanG első pillantásra szembetűnő jellegzetessége, hogy kulcsszavai magyar nyelvűek, ami magyar diákok számára növelheti a kulcsszavak szerepkifejező erejét.
Mégis sok kulcsszó esetlen, furcsa megfogalmazású (\plang{MEGNYIT}, \plang{KI}, \plang{KEREK}, \plang{RND}).
Míg az angol programozási nyelvek kulcsszavai, függvénynevei rendszerint felszólító módú igéks, addig a PLanG módszeresen főneveket használ, még a függvénynevekre is.
Az ilyen függvények, utasítások szerepkifejező képessége alacsony (vajon \plang{KEREK 10.5} kerekíti az értéket, vagy azt mondja meg, kerek szám-e az érték? a \plang{NAGY 'z'} azt mondja, nagybetű-e a paraméter vagy nagybetűvé alakít?).
A kulcsszavak közül a \plang{CIKLUS} nem feltétlen érthető egy olyan diáknak, aki még soha nem programozott; szerepkifejezőbb lehetett volna például az *\plang{ISMÉTELD} kulcsszó.

Az operátorok között is vannak alacsony szerepkifejezésű elemek: nem magától értetődő például, hogy a \plang{/=} az áthúzott egyenlőségjel helyett áll, és a \plang{DIV} és a \plang{/} osztások közül sem evidens, melyik melyik.
Hasonlóképp, bár kis programozói előképzettséggel ,,nyilvánvaló'' a \plang{:=} értékadó utasítás szemantikája, de kezdőknek nem feltétlen az: anekdotikus bizonyíték szerint viszonylag gyakori hiba, hogy a diák nem tudja, melyik oldal adja, és melyik kapja az értéket.
Valóban, a \plang{:=} csak az Algol nyelvben szerepelt először értékadásként, és jó példája a \textit{memetikus konzisztenciának}, vagyis hogy csak azért csinálunk valamit úgy, mert mások is úgy csinálták, és nem vesszük figyelembe az eltérő igényeket\cite{McIver01}.
Szintén alacsony a tömbdeklarációk szerepkifejező ereje (\plang{EGÉSZ[5]}), és ez is a memetikus konzisztenciából fakad.

A leképezés közelisége közepes-alacsony.
Jól teljesítenek az operátorok, amelyek viselkedésükben és megjelenésükben is megfelelnek a diákok matematikai tudásának.
Különösen szép, hogy az unáris függvényeket nem kell zárójelezni (\plang{sin x}), és egyedi, ám nagyszerű az elemszám-lekérdezés cirkumfix operátora (\plang{|tomb|}).
%, bár zavaró lehet a matematikai jelölésben megszokott *\plang{a < b < c} jellegű összehasonlítás hiánya.
A tömbök viszont már nagyon távol vannak a hallgató gondolkodásától: méretük mindig fix, alapműveleteik hiányoznak.
Lehet persze avval érvelni, hogy például a tartalmazásvizsgálat megvalósítása a hallgató feladata, így helyes, hogy a PLanGban nincs ilyen művelet -- azonban az absztrakciós eszközök teljes hiánya a hallgatót arra kényszeríti, hogy minden esetben manuálisan, ciklussal végezze el a tartalmazásvizsgálatot, ami csökkenti a robosztusságot és növeli a szellemi erőfeszítést.
Ismerősebb lenne a tanulóknak egy, a matematikai halmazokhoz hasonló összetett típus.

A nyelvtan szép, letisztult. Nagy erény, hogy nincs szükség utasításlezáró jelre, ez nagyban csökkenti a ,,becsúszó'' hibák valószínűségét.

A PLanG egyáltalán nem nyújt absztrakciós eszközöket, a programozó semmilyen formában nem változtathat a nyelven.
Az alprogramok hiánya nagyban növeli a terjengősséget és a szellemi erőfeszítést, a felhasználó által definiált típusok hiánya csökkenti a robosztusságot és a módszertani helyességet.






% \subsection{tervezési céloks és döntések}
% szintaxis és szemantika szeparációja

%..
% java

% \subsection{Célok megvalósulása/postmortem}
% túlzott generalizáció
% java
% more is more error in langdesign

%Célszerű lett volna kérdőíves felmérést végezni elsőéves diákok részvételével, komolyabban kutatni a nyelvek pedagógiai aspektusait, illetve elemezni lehetett volna az egyetem rendelkezésére álló nagyméretű korpuszt; erre viszont erőforrások hiányában nem került sor.
